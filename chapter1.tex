\section{Chapter One}

\exercise %1.1
Explain in one paragraph at least three levels
of abstraction that are used by
\begin{tasks}(2)
	\task biologists studying the operation of cells
	\task chemists studying the composition of matter	
\end{tasks}
\solution
\begin{enumerate}[label=\alph*)]
\item Cells consist of \emph{atoms}. Atoms form special molecules known as 
\emph{amino acids}. These are used by the cell to create \emph{enzymes}. 
\item There are a lot of \emph{quarks}, which (in threes) form basic particles
such as neutrons, electrons, protons, anti-protons, positrons. If you then
take a number of neutrons, electrons and protons, you get an atom.
\end{enumerate}

\exercise %1.2
Explain in one paragraph how the techniques of hiearchy, modularity,
and regularity may be used by
\begin{tasks}(2)
	\task automobile designers
	\task businesses to manage their operations
\end{tasks}
\solution
\todo{1.2}

\exercise %1.3
Ben Bitdiddle is building a house. Explain how he can use the principle
of \emph{hiearchy}, \emph{modularity}, and \emph{regularity} to save time 
and money during construction
\solution
First, let's define the terms used here:
\begin{description}
	\item[Hiearchy] dividing a system into modules
	\item[Modularity] modules can be used without having to understand
	how they work inside
	\item[Regularity] modules can be reused easily
\end{description}
\todo{1.3}

\exercise
An analog voltage is in the range of 0-5$V$. If it can be measured with
an accuracy of $\pm 50mV$, at most how many bits of information does it convey?
\solution
If the analog voltage can be measured with an accuracy of $\pm 50mV$, then
it can only accurately be measured to the nearest $100mV$. This means that
between $0mV$ and $5,000mV$, it can take on 50 differentiable
values:
\begin{equation*}
\frac{5,000\mathrm{mV}}{100\mathrm{mV}}=50	
\end{equation*}
Since you need $log_2(n)$ bits to be able to differentiate between $n$
Values, the analog voltage has $log_2(50) \approx 5.64$ bits of information, 
so it conveys at most 6 bits of information.

\exercise
A classroom has an old clock on the wall whose minute hand broke off.
\begin{tasks}
	\task If you can read the hour hand to the nearest 15 minutes, how many
	bits of information does the clock convey about time?
	\task If you know whether it is before or after noon, how many additional
	bits of information do you know about the time?	
\end{tasks}
\solution
\begin{enumerate}[label=\alph*)]
	\item An hour has 60 minutes, if you know the time to the nearest
	minute that makes 4 different states. Then there are 12 hours per day,
	which makes a total of $12*4=48$ different states. Using $log_2(n)$,
	that makes $log_2(48) \approx 5.58$ bits of information.
	\item If you know whether it is before or after noon, that makes two
	states, and that is exactly one bit. So by knowing that, you know
	one additional bit about the time.
\end{enumerate}

\exercise
The babylonians developed the \emph{sexagesimal} (base 60) number system
about 4000 years ago. How many bits of information is conveyed with one
sexagesimal digit? How do you write the number $4000_{10}$ in sexagesimal?
\solution
Just like we have ten digits ($0..9$) in base 10, they had sixty digits in
their number system. Since there are sixty different digit, one single one
of them conveys $log_2(60) \approx \mathbf{5.91}$ bits of information. 
Converting numbers can be done just like in any other number system:
\begin{align*}
4000 / 60 & = 66 \: \mathtt{rem} \: 40 \\
66 / 60 & = 1 \: \mathtt{rem} \: 6 \\
1 / 60 & = 0 \: \mathtt{rem} \: 1	
\end{align*}
By reading the remainders off, the number is: $1 \cdot 60^2+6 \cdot 60^1+40
\cdot 60^0$.

\exercise
How many different numbers can be represented with 16 bits?
\solution
$2^{16}=65536$ numbers can be represented with 16 bits.

\exercise
What is the largest unsigned 32-bit number?
\solution
$2^{32}-1=4,294,967,295$, because there are $2^{32}=4,294,967,295$ unsigned
32-bit numbers and the first one is $0$.

\exercise
What is the largest 16-bit binary number that can be represented with 
\begin{tasks}(3)
	\task unsigned numbers?
	\task two's complement numbers?
	\task sign/magnitude numbers?
\end{tasks}
\solution
\begin{tasks}
	\task There are $2^{16}=65\,536$ unsigned 16-bit numbers, the 
	largest being $2^{16}-1=\mathbf{65\,535}$.
	\task There are $2^{16}=65\,536$ 16-bit numbers in two's complement form,
	the biggest is $2^{15}-1=\mathbf{32\,767}$.
	\task There are $2^{16}=65\,536$ 16-bit numbers in sign/magnitude form,
	the biggest being
	$2^{15}-1=\mathbf{32\,767}$.
\end{tasks}

\exercise
What is the largest 32-bit number that can be represented with 
\begin{tasks}(3)
	\task unsigned numbers?
	\task two's complement numbers?
	\task sign/magnitude numbers?
\end{tasks}
\solution
\begin{tasks}
	\task Largest 32-bit unsigned number is $2^{32}-1=\mathbf{4\,294\,967\,295}$.
	\task Largest 32-bit two's complement number is $2^{31}-1=\mathbf{2\,147\,483\,648}$.
	\task Largest 32-bit sign/magnitude number is $2^{31}-1=\mathbf{2\,147\,483\,648}$.
\end{tasks}

\exercise
What is the smallest (most negative) 16-bit binary number that can be
represented with
\begin{tasks}(3)
	\task unsigned numbers?
	\task two's complement numbers?
	\task sign/magnitude numbers?
\end{tasks}
\solution
\begin{tasks}
	\task Smallest 16-bit unsigned number is \textbf{0}.
	\task Smallest 16-bit two's complement number is $-2^{15}=\mathbf{-32\,768}$.
	\task Smallest 16-bit sign/magnitude number is $-2^{15}+1=\mathbf{-32\,767}$.
\end{tasks}

\exercise
What is the smallest (most negative) 32-bit binary number that can be
represented with 
\begin{tasks}(3)
	\task unsigned numbers?
	\task two's complement numbers?
	\task sign/magnitude numbers?
\end{tasks}
\solution
\begin{tasks}
	\task Smallest 32-bit unsigned number is \textbf{0}.
	\task Smallest 32-bit two's complement number is $-2^{31}=\mathbf{-2\,147\,483\,648}$.
	\task Smallest 32-bit sign/magnitude number is $-2^{31}+1=\mathbf{2\,147\,483\,647}$.
\end{tasks}

\exercise %1.13
Convert the following unsigned binary numbers to decimal. Show your work.
\begin{tasks}(4)
	\task $1010_2$
	\task $11\,0110_2$
	\task $1111\,0000_2$
	\task $000\,1000\,1010\,0111_2$
\end{tasks}
\solution
\begin{tasks}
	\task $1010_2=1 \cdot 2^3 + 0 \cdot 2^2 + 1 \cdot 2^1 + 0 \cdot 2^0 
	= 8 + 2 = \mathbf{10}$
	\task $110110_2 = 1 \cdot 2^5 + 1 \cdot 2^4 + 0 \cdot 2^3 + 1 \cdot 2^2
	+ 1 \cdot 2^1 + 0 \cdot 2^0 = 32+ 16 + 4 + 2 = \mathbf{54}$
	\task $11110000_2 = 1 \cdot 2^7 + 1 \cdot 2^6 + 1 \cdot 2^5 + 1 \cdot 2^4
	+ 0 \cdot 2^3 + 0 \cdot 2^2 + 0 \cdot 2^1 + 0 \cdot 2^0 = 128+64+32+16
	= \mathbf{240}$
	\task $000100010100111_2 = 2^11 + 2^7 + 2^5 + 2^2 + 2^1 + 2^0
	= 2048 + 256 + 32 + 4 + 2 + 1 = \mathbf{2\,343}$
\end{tasks}

\exercise %1.14
Convert the following unsigned binary numbers to decimal. Show your work.
\begin{tasks}(4)
	\task $1110_2$
	\task $10\,0100_2$
	\task $1101\,0111_2$
	\task $011\,1010\,1010\,0100_2$
\end{tasks}
\solution
\begin{tasks}
	\task $1110_2 = 2^3 + 2^2 + 2^1 = 8 + 4 + 2 = \mathbf{14}$
	\task $100100_2 = 2^5 + 2^2 = 32 + 4 = \mathbf{36}$
	\task $11010111_2 = 2^7 +2^6+2^4+2^2+2^1 + 2^0 = 128+64+16+4+2+1=\mathbf{215}$
	\task $011101010100100_2 = 2^13+2^12+2^11+2^9+2^7+2^5+ 2^2=8192+
	4096+2048+512+128+32+4=\mathbf{15\,012}$
\end{tasks}

\exercise %1.15
Repeat Exercise 1.13, but convert to hexadecimal.
\solution
\begin{tasks}
	\task $1010_2=1010_2 \cdot 16^0 = \mathbf{A_{16}}$
	\task $110110_2 = 11_2 \cdot 16^1 + 0110_2 \cdot 16^0 = \mathbf{36_{16}}$
	\task $11110000_2 = 1111_2 \cdot 16^1 + 0000_2 \cdot 16^0 = 
	\mathbf{F0_{16}}$
	\task $000100010100111_2 = 000_2 \cdot 16^3 + 1000_2 \cdot 16^2 +
	1010_2 \cdot 16^1 + 0111_2 \cdot 16^0 = \mathbf{8\,A7_{16}}$
\end{tasks}

\exercise %1.16
Repeat Exercise 1.14, but convert to hexadecimal.
\solution
\begin{tasks}
	\task $1110_2 = \mathbf{E_{16}}$
	\task $100100_2 = \mathbf{24_{16}}$
	\task $11010111_2 = \mathbf{D7_{16}}$
	\task $011101010100100_2 = \mathbf{3AA4_{16}}$
\end{tasks}

\exercise %1.17
Convert the following hexadecimal numbers to decimal. Show your work.
\begin{tasks}(4)
	\task $\mathrm{A5}_{16}$
	\task $\mathrm{3B}_{16}$
	\task $\mathrm{FF\,FF}_{16}$
	\task $\mathrm{D0}\,00\,00\,00_{16}$
\end{tasks}
\solution
\begin{tasks}
	\task $\mathrm{A5}_{16} = 10\cdot 16^1 + 5\cdot 16^0 = 160+5 = \mathbf{165}$
	\task $\mathrm{3B}_{16} = 3\cdot 16^1 + 11\cdot 16^0 = 48+11=\mathbf{59}$
	\task $\mathrm{FFFF}_{16} = 15\cdot 16^3 + 15\cdot 16^2 + 15\cdot 16^1 +
	15\cdot 16^0= 61440 + 3840 + 240 + 15 = \mathbf{65\,535}$
	\task $\mathrm{D0000000}_{16} = 13 \cdot 16^7 = \mathbf{3\,489\,660\,928}$
\end{tasks}

\exercise %1.18
Convert the following hexadecimal numbers to decimal. Show your work.
\begin{tasks}(4)
	\task $\mathrm{4E_{16}}$
	\task $\mathrm{7C_{16}}$
	\task $\mathrm{ED\,3A_{16}}$
	\task $\mathrm{40\,3F\,B0\,01_{16}}$
\end{tasks}
\solution
\begin{tasks}
	\task $\mathrm{4E}_{16} = 4\cdot 16^1 + 14\cdot 16^0 = 64 + 14 = \mathbf{78}$
	\task $\mathrm{7C}_{16} = 7\cdot 16^1 + 12\cdot 16^0 = 112 + 12 = \mathbf{124}$
	\task $\mathrm{ED3A}_{16} = 14\cdot 16^3 + 13\cdot 16^2 + 3\cdot 16^1
	+ 10 \cdot 16^0 = 57355 + 3328 + 48 + 10 = \mathbf{60\,741}$
	\task $\mathrm{403FB001}_{16} = 4\cdot 16^7 + 0\cdot 16^6 + 3\cdot 16^5
	+ 15\cdot 16^4 + 11\cdot 16^3 + 0\cdot 16^2 + 0\cdot 16^1 + 1\cdot 16^0\\
	\phantom{\mathrm{403FB001}_{16}}
	= 1073741824 + 3145728 + 983040 + 45056 + 1 \\
	\phantom{\mathrm{403FB001}_{16}} = \mathbf{1\,077\,915\,649}$
\end{tasks}

\exercise
Repeat Exercise 1.17, but convert to unsigned binary.
\solution
\begin{tasks}
	\task $\mathrm{A5}_{16} = 10\cdot 16^1 + 5\cdot 16^0 = 1010_2 \cdot 16^1 + 
	0101_2\cdot 16^0 = \mathbf{1010\,0101_2}$
	\task $\mathrm{3B}_{16} = 3\cdot 16^1 + 11\cdot 16^0 = 0011_2 \cdot 16^1 +
	1011_2 \cdot 16^0 = \mathbf{0011\,1011_2}$
	\task $\mathrm{FFFF}_{16} = 15\cdot 16^3 + 15\cdot 16^2 + 15\cdot 16^1 +
	15 \cdot 16^0 \\
	\phantom{\mathrm{FFFF}_{16}} = 1111_2 \cdot 16^3 + 1111_2 \cdot 16^2 + 
	1111_2 \cdot 16^1 + 1111_2 \cdot 16^0 \\
	\phantom{\mathrm{FFFF}_{16}} = \mathbf{1111\,1111\,1111\,1111_2}$
	\task $\mathrm{D0000000}_{16} = 13*16^7 = 1101_2 \cdot 16^7 = 
	\mathbf{1101\,0000\,0000\,0000\,0000\,0000\,0000\,0000_2}$
\end{tasks}

\exercise
Repeat Exercise 1.18, but convert to unsigned binary.
\solution
\begin{tasks}
	\task $\mathrm{4E}_{16} = 4\cdot 16^1 + 14\cdot 16^0 = 0100_2 \cdot 16^1
	+ 1110_2 \cdot 16^0 = \mathbf{0100\,1110_2}$
	\task $\mathrm{7C}_{16} = 7\cdot 16^1 + 12\cdot 16^0 = 0111_2 \cdot 16^1
	+ 1100_2 \cdot 16^0 = \mathbf{0111\,1100_2}$
	\task $\mathrm{ED3A}_{16} = 14\cdot 16^3 + 13\cdot 16^2 + 3 \cdot 16^1 +
	10\cdot 16^0\\
	\phantom{\mathrm{ED3A}_{16}}
	= 1101_2 \cdot 16^3 + 1100_2 \cdot 16^2 + 0011_2 \cdot 16^1
	+ 1010_2 \cdot 16^0\\
	\phantom{\mathrm{ED3A}_{16}} = \mathbf{1101\,1100\,0011\,1010_2}$
	\task $\mathrm{403FB001}_{16} = 4\cdot 16^7 + 0\cdot 16^6 + 3\cdot 16^5
	+ 15\cdot 16^4 + 11\cdot 16^3 + 0\cdot 16^2 + 0\cdot 16^1 + 1\cdot 16^0 \\
	\phantom{\mathrm{403FB001}_{16}} = 0100_2\cdot 16^7 + 0000_2\cdot 16^6 +
	 0011_2\cdot 16^5 + 1111_2\cdot 16^4 + 1011_2\cdot 16^3 + \\
	\phantom{\mathrm{403FB001}_{16} =} \: 0000_2\cdot 16^2 + 0000_2\cdot 16^1 + 
	0001_2\cdot 16^0 \\
	\phantom{\mathrm{403FB001}_{16}} = 
	\mathbf{0100\,0000\,0011\,1111\,1011\,0000\,0000\,0001_2}$
\end{tasks}

\exercise
Convert the following two's complement binary numbers to decimal.
\begin{tasks}(4)
	\task $1010_2$
	\task $11\,0110_2$
	\task $0111\,0000_2$
	\task $1001\,1111_2$
\end{tasks}
\solution
\begin{tasks}
	\task $1010_2 = 1 \cdot (-2^3) + 0 \cdot 2^2 + 1 \cdot 2^1 + 0 \cdot 2^0
	= -8 + 2 = \mathbf{-6}$
	\task $110110_2 = 1 \cdot (-2^5) + 1 \cdot 2^4 + 1 \cdot 2^2 + 1 \cdot 2^1
	= -32 + 16 + 4 + 2 = \mathbf{-10}$
	\task $01110000_2 = 1 \cdot 2^6 + 1 \cdot 2^5 + 1 \cdot 2^4 = 
	64 + 32+  16 = \mathbf{112}$
	\task $10011111_2 = 1 \cdot (-2^7) + 1 \cdot 2^4 + 1 \cdot
	2^3 + 1 \cdot 2^2 + 1 \cdot 2^1 + 1 \cdot 2^0
	= -128 + 16+8+4+2+1 = \mathbf{-97}$	
\end{tasks}

\exercise %1.22
Convert the following two's complement binary numbers to decimal.
\begin{tasks}(4)
	\task $1110_2$
	\task $10\,0011_2$
	\task $0100\,1110_2$
	\task $1011\,0101_2$
\end{tasks}
\solution
\begin{tasks}
	\task $1110_2 = -2^3 + 2^2 + 2^1 = -8 + 4 + 2 = \mathbf{-2}$
	\task $10\,0011_2 = -2^5 + 2^1 + 2^0 = -32 + 2 + 1 = \mathbf{-29}$
	\task $0100\,1110_2 = 2^6 + 2^3 + 2^2 +  2^1 = 64+8+4+2 = \mathbf{78}$
	\task $1011\,0101_2 = -2^7 + 2^5 + 2^4 + 2^2 + 2^0 = -128 + 32+16+4+1=\mathbf{-75}$
\end{tasks}

\exercise %1.23
Repeat Exercise 1.21, assuming the binary numbers are in sign/magnitude
form rather than two's complement representation.
\solution
\todo{1.23}

\exercise %1.24
Repeat Exercise 1.22, assuming the binary numbers are in sign/magnitude
form rather than two's complement representation.
\solution
\todo{1.24}

\exercise %1.25
Convert the following decimal numbers to unsigned
binary numbers.
\begin{tasks}(4)
	\task $42_{10}$
	\task $63_{10}$
	\task $229_{10}$
	\task $845_{10}$	
\end{tasks}
\solution
\begin{tasks}
	\task $42=32+8+2=2^5+2^3+2^1=\mathbf{10\,1010_2}$	
	\task $63=32+16+8+4+2+1=2^5+2^4+2^3+2^2+2^1+2^0=\mathbf{11\,1111_2}$
	\task $229=128+64+32+4+1=2^7+2^6+2^5+2^2+2^0=\mathbf{1110\,0101_2}$
	\task $845=512-256-64-8-4-1=2^9+2^8+2^6+2^3+2^2+2^0=\mathbf{11\,0100\,1101_2}$
\end{tasks}

\exercise %1.26
Convert the following decimal numbers to unsigned binary numbers.
\begin{tasks}(4)
	\task $14_{10}$
	\task $52_{10}$
	\task $339_{10}$
	\task $711_{10}$
\end{tasks}

\exercise %1.27
Repeat Exercise 1.25, but convert to hexadecimal.

\exercise %1.28
Repeat Exercise 1.26, but convert to hexadecimal.

\exercise %1.29
Convert the following decimal numbers to 8-bit two's complement numbers or
indicate that the decimal number would overflow the range.
\begin{tasks}(5)
	\task $42_{10}$
	\task $-63_{10}$
	\task $124_{10}$
	\task $-128_{10}$
	\task $133_{10}$
\end{tasks}

\exercise %1.30
Convert the following decimal numbers to 8-bit two's complement numbers or
indicate that the decimal number would overflow the range.
\begin{tasks}(5)
	\task $24_{10}$
	\task $-59_{10}$
	\task $128_{10}$
	\task $-150_{10}$
	\task $127_{10}$
\end{tasks}

\exercise %1.31
Repeat Exercise 1.29, but convert to 8-bit sign/magnitude numbers.

\exercise %1.32
Repeat Exercise 1.30, but convert to 8-bit sign/magnitude numbers.

\exercise %1.33
Convert the following 4-bit two's complement numbers to 8-bit two's complement numbers.
\begin{tasks}
	\task $0101_2$
	\task $1010_2$
\end{tasks}

\exercise %1.34
Convert the following 4-bit two's complement numbers to 8-bit two's complement numbers.
\begin{tasks}
	\task $0111_2$
	\task $1000_2$
\end{tasks}

\exercise %1.35
Repeat Exercise 1.33 if the numbers are unsigned rather than two's complement

\exercise %1.36
Repeat Exercise 1.34 if the numbers are unsigned rather than two's complement

\exercise %1.37
Base 8 is referred to as octal. Convert each of the numbers from Exercise 1.25 to octal.

\exercise %1.38
Base 8 is referred to as octal. Convert each of the numbers from Exercise 1.26 to octal.

\exercise %1.39
Convert each of the following octal numbers to binary, hexadecimal, and decimal.
\begin{tasks}
	\task $42_8$
	\task $63_8$
	\task $255_8$
	\task $3047_8$
\end{tasks}

\exercise %1.40
Convert each of the following octal numbers to binary, hexadecimal, and decimal.
\begin{tasks}
	\task $23_8$
	\task $45_8$
	\task $371_8$
	\task $2560_8$
\end{tasks}

\exercise %1.41
How many 5-bit two's complement numbers are greater than 0? How many are less than 0?
How would your answers differ for sign/magnitude numbers?

\exercise %1.42
How many 7-bit two's complement numbers are greater than 0? How many are less than 0?
How would your answers differ for sign/magnitude numbers?

\exercise %1.43
How many bytes are in a 32-bit word? How many nibbles are in the word?

\exercise %1.44
How many bytes are in a 64-bit word?

\exercise %1.45
A particular DSL modem operates at 768 kbits/sec. How many bytes can it receive in 1 minute?

\exercise %1.46
USB 3.0 can send data at 5 Gbits/sec. How many bytes can it send in 1 minute?

\exercise %1.47
Hard disk manufacturers use the term "megabytes" to mean $10^6$ bytes and "gigabytes" to mean
$10^9$ bytes. How many real GBs of music can you store on a 50 GB hard disk?

\exercise %1.48
Estimate the value of $2^31$ without using a calculator.

\exercise %1.49
A memory on the Pentium 2 microprocessor is organized as a rectangular array of bits with $2^8$
rows and $2^9$ columns. Estimate how many bits it has without using a calculator.

\exercise %1.50
Draw a number line analogous to Figure 1.11 for 3-bit unsigned, two's complement,
and sign/magnitude numbers.

\exercise %1.51
Draw a number line analogous to Figure 1.11 for 2-bit unsigned, two's complement,
and sign/magnitude numbers.

\exercise %1.52
Perform the following additions of unsigned binary numbers. Indicate whether or not the sum
overflows a 4-bit result.
\begin{tasks}
	\task $1001_2 + 0100_2$
	\task $1101_2 + 1011_2$
\end{tasks}

\exercise %1.53
Perform the following additions of unsigned binary numbers. Indicate whether or not the sum
overflows a 8-bit result.
\begin{tasks}
	\task $10011001_2 + 01000100_2$
	\task $11010010_2 + 10110110_2$
\end{tasks}

\exercise %1.54
Repeat Exercise 1.52, assuming that the binary numbers are in two's complement form.

\exercise %1.55
Repeat Exercise 1.53, assuming that the binary numbers are in two's complement form.

\exercise %1.56
Convert the following decimal numbers to 6-bit two's complement binary numbers and add them.
Indicate whether or not the sum overflows a 6-bit result.
\begin{tasks}
	\task $16_{10} + 9_{10}$
	\task $27_{10} + 31_{10}$
	\task $-4_{10} + 19_{10}$
	\task $3_{10} + -32_{10}$
	\task $-16_{10} + -9_{10}$
	\task $-27_{10} + -31_{10}$
\end{tasks}

\exercise %1.57
Repeat Exercise 1.56 for the following numbers.
\begin{tasks}
	\task $7_{10} + 13_{10}$
	\task $17_{10} + 25_{10}$
	\task $-26_{10} + 8_{10}$
	\task $31_{10} + -14_{10}$
	\task $-19_{10} + -22_{10}$
	\task $-2_{10} + -29_{10}$
\end{tasks}

\exercise %1.58
Perform the following additions of unsigned hexadecimal numbers. Indicate whether or not the
sum overflows an 8-bit (two hex digit) result.
\begin{tasks}
	\task $7_{16} + 9_{16}$
	\task $13_{16} + 28_{16}$
	\task $AB_{16} + 3E_{16}$
	\task $8F_{16} + AD_{16}$
\end{tasks}

\exercise %1.59
Perform the following additions of unsigned hexadecimal numbers. Indicate whether or not the
sum overflows an 8-bit (two hex digit) result.
\begin{tasks}
	\task $22_{16} + 8_{16}$
	\task $73_{16} + 2C_{16}$
	\task $7F_{16} + 7F_{16}$
	\task $C2_{16} + A4_{16}$
\end{tasks}

\exercise %1.60
Convert the following decimal numbers to 5-bit two's complement binary numbers and subtract them.
Indicate whether or not the difference overflows a 5-bit result.
\begin{tasks}
	\task $9_{10} - 7_{10}$
	\task $12_{10} - 15_{10}$
	\task $-6_{10} - 11_{10}$
	\task $4_{10} - - 8_{10}$
\end{tasks}


%\todo{1.60}
%\todo{1.61}
%\todo{1.62}
%\todo{1.63}
%\todo{1.64}
%\todo{1.65}
%\todo{1.66}
%\todo{1.67}
%\todo{1.68}
%\todo{1.69}
%\todo{1.70}
%\todo{1.71}
%\todo{1.72}
%\todo{1.73}
%\todo{1.74}
%\todo{1.75}
%\todo{1.76}
%\todo{1.77}
%\todo{1.78}
%\todo{1.79}
%\todo{1.80}
%\todo{1.81}
%\todo{1.82}
%\todo{1.83}
%\todo{1.84}
%\todo{1.85}
%\todo{1.86}
%\todo{1.87}
%\todo{1.88}
%\todo{1.89}
%\todo{1.90}
